\documentclass[a4paper,11pt]{article}

\usepackage[T1]{polski}
\usepackage[utf8]{inputenc}
\usepackage{verbatim}
\usepackage{graphics}
\usepackage{graphicx}
\usepackage{csvsimple}
\usepackage[figurename=Zrzut\ ekranu]{caption}

\hoffset=-3.0cm                         
\textwidth=18cm                         
\evensidemargin=0pt

\voffset=-3cm                           
\textheight=27cm                        

\setlength{\parindent}{0pt}             
\setlength{\parskip}{\medskipamount}    
\raggedbottom              

\title{Postępy prac projktu indywidualnego - cz. 2}
\author{Michał Banaszczak}
\date{4 czerwca 2022} 

\begin{document}

\maketitle

\section{Napotkane problemy}
\subsection{Brak spójności nazw państw uczestniczących i przypisów}
Na poszczególnych podstronach z niewiadomych na pierwszy rzut oka zamiast nazw
państw uczestniczących figurują trzyliterowe kody. Te z kolei nie są oficjalnymi
kodami państw, a wymyślonymi specjalnie na potrzeby IO (np. \textit{SUI} dla
Szwajcarji zamiast oficjalnego \textit{CHE}). Dodatkowo w IO brały udział państwa
już nie istniejące lub różne sportowe organizacje pozapaństwowe. Z tego względu
jedynym wyjściem ujednolicenia zbioru było zmapowanie wszystkich występujących
kodów i zamiana ich na pełne nazwy państw lub organizacji. Analogicznie zastąpiono
puste wyniki przypisami wyjaśniającymi ich brak.

\subsection{Czas scrapowania}
Ilość danych do pobrania i zapisania wraz z wyżej opisaną niespójnością danych
sprawiały, że jakiekolwiek formatowanie zbioru w trakcie pobierania okazało się
praktycznie niemożliwe, rzucając cały czas błędami. Z tego względu pobieranie 
i formatowanie podzielono na dwa moduły: \verb|scoreRetriever.js| służył do
pobrania surowych danych, a \verb|scoreFormatter.js| - do ich formatowania.
Pozwoliło to na sprawniejsze naprawianie błędów wynikających z niespójności danych
na bieżąco. Moduł pobierający dodatkowo może pobierać dane fragmentami, dzięki
czemu przy jakimkolwiek błędzie nie traci się dotąd zapisanych danych. 

\section{Postępy}
\subsection{Zbiór linków z danymi z poszczególnych dyscyplin sportowych}
Po wykonaniu wyżej opisango scrapowania otrzymano zbiór wszystkich statycznych


\end{document}